%%%%%%%%%%%%%%%%%%%%%%%%%%%%%
%% Styles, packages and new commands
\input{../../Main/ML_Main.tex}
%%%%%%%%%%%%%%%%%%%%%%%%%%%%%
%% Edit the title page
\title{Machine Learning}
\subtitle{Module 3.0 - Models: Introduction}
\author[MOB]{Marc-Olivier Boldi}
\institute[HEC MSc Mgt BA]{Master in Management, Business Analytics, HEC UNIL}
\date[Spring 2024]{Spring 2024}
%%%%%%%%%%%%%%%%%%%%%%%%%%%%%
%%%%%%%%%%%%%%%%%%%%%%%%%%%%%
%%%%%%%%%%%%%%%%%%%%%%%%%%%%%
%%%%%%%%%%%%%%%%%%%%%%%%%%%%%
\begin{document}
%%%%%%%%%%%%%%%%%%%%%%%%%%%%%
\begin{frame}
  \titlepage
\end{frame}
%%%%%%%%%%%%%%%%%%%%%%%%%%%%%
%%%%%%%%%%%%%%%%%%%%%%%%%%%%%
\begin{frame}
\frametitle{Context}
In ML, models are mainly used for {\bf supervised learning}, the aim is
\begin{itemize}
\item Predict a response $y$: regression if numerical, classification if categorical.
\item From features $x=\{x_1, \ldots, x_p\}$: available at the moment of prediction,
\item With the best possible quality: built from the data in an optimal way.
\end{itemize}
The $n$ observed features and responses are denoted 
$$
(y_1, x_1), \ldots, (y_n, x_n).
$$
\end{frame}
%%%%%%%%%%%%%%%%%%%%%%%%%%%%%
\begin{frame}
\frametitle{Elements}
In ML, a model consists mainly of three elements:
\begin{itemize}
\item A prediction formula, taking the features $x$, returning a prediction $\hat{y} = f(x)$ for $y$,
\item A loss function ${\cal L}(y,\hat{y})$ measuring how "wrong" a prediction $\hat{y}$ is for $y$.
\item An algorithm which can optimize the prediction formula $f$ using the observed data.
\end{itemize}
\end{frame}
%%%%%%%%%%%%%%%%%%%%%%%%%%%%%
\begin{frame}
\frametitle{The prediction formula}
The prediction formula is a mathematical formula (sometimes a more complex algorithm) using {\bf parameters}\footnote{Also called {\bf weights}, especially for Neural Networks.} $\theta$, combining them with the feature $x$, returning a prediction
$$
f(x;\theta).
$$
Thus, $\theta$ must be chosen carefully to obtain good predictions of $y$.
\end{frame}
%%%%%%%%%%%%%%%%%%%%%%%%%%%%%
\begin{frame}
\frametitle{The loss function}
The loss function indicates how wrong is a prediction $\hat{y}$ of the corresponding $y$.\\
\vspace{0.3cm} 
A classical example for regression is the square of the error:
$$
{\cal L}(y,\hat{y}) = (y-\hat{y})^2.
$$ 
The larger ${\cal L}(y,\hat{y})$, the further $\hat{y}$ is from $y$.
\end{frame}
%%%%%%%%%%%%%%%%%%%%%%%%%%%%%
\begin{frame}
\frametitle{The optimal parameters}
Good parameters $\theta$ must have a low loss. We want ${\cal L}(y,f(x;\theta))$ to be small for all $(y,x)$. To achieve an overall quality on the whole available data base, we want $\theta$ achieving a small
$$
\bar{{\cal L}}(\theta) = \frac{1}{n}\sum_{i=1}^n {\cal L}\{y_i, f(x_i;\theta)\}.
$$
Example: with the square of the error, this is 
$$
\bar{{\cal L}}(\theta) = \frac{1}{n}\sum_{i=1}^n \{y_i, f(x_i-\theta)\}^2.
$$
\end{frame}
%%%%%%%%%%%%%%%%%%%%%%%%%%%%%
\begin{frame}
\frametitle{The optimization algorithm}
Finding the optimal $\hat{\theta}$ is done by applying an algorithm, i.e., a procedure that finds 
$$
\hat{\theta} = \arg\min_{\theta} \bar{{\cal L}}(\theta).
$$
The algorithm is often a sequential procedure. It builds a sequence $\theta_1, \theta_2, \theta_3, \ldots$ such that
$$
\bar{{\cal L}}(\theta_1) > \bar{{\cal L}}(\theta_2) > \bar{{\cal L}}(\theta_3) > \ldots 
$$
Ultimately, this should reach the minimum possible $\bar{{\cal L}}(\theta)$.
\end{frame}
%%%%%%%%%%%%%%%%%%%%%%%%%%%%%
\begin{frame}
\frametitle{Mathematical considerations}
\small
\begin{itemize}
\item More flexible model $f$ provides better opportunity to minimize $\bar{{\cal L}}$. Often, this is associated with the size of $\theta$ (number of parameters).
\item The algorithm may not reach the global minimum of $\bar{{\cal L}}$. Most algorithms cannot guaranty such results except under theoretical assumptions.
\item A probabilistic interpretation: the optimal $\theta$ is obtained by minimizing the expected loss on the population of $(Y,X)$
$$
E\left[{\cal L}\{Y, f(X;\theta)\}\right].
$$
The data base is used to estimate it with an empirical mean
$$
\hat{E}\left[{\cal L}\{Y, f(X;\theta)\}\right] = \frac{1}{n}\sum_{i=1}^n {\cal L}\{y_i, f(x_i;\theta)\}.
$$
This estimate is minimized in turn to find an estimate of the optimal $\theta$.
\end{itemize}
\normalsize
\end{frame}
%%%%%%%%%%%%%%%%%%%%%%%%%%%%%
\end{document}

