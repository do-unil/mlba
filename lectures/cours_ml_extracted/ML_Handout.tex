\documentclass[twocolumn]{article}
\usepackage{amsmath, amssymb, graphicx, hyperref}
\title{Machine Learning: Linear and Logistic Regressions}
\author{Marc-Olivier Boldi\\Master in Management, Business Analytics, HEC UNIL\\Spring 2024}
\date{}

\begin{document}

\maketitle

\section{Linear Regression}
Linear regression is a fundamental supervised learning technique used for predicting a continuous outcome variable based on one or more predictor variables. It assumes a linear relationship between the input features and the target variable.

\subsection{Prediction Formula}
The linear regression model can be expressed as:
\begin{equation}
    f(x; \theta) = \theta_0 + \theta_1 x_1 + \cdots + \theta_p x_p
\end{equation}
where:
\begin{itemize}
    \item $\theta_0$ is the intercept,
    \item $\theta_1, \dots, \theta_p$ are the coefficients of the model,
    \item $x_1, \dots, x_p$ are the feature values.
\end{itemize}

\textbf{Example:} Suppose we have a dataset with one feature $x$ representing the number of study hours and $y$ representing exam scores. The regression equation might look like:
\begin{equation}
    y = 50 + 10x
\end{equation}
This implies that for each additional hour of study, the expected exam score increases by 10 points.

\subsection{Loss Function}
The most commonly used loss function for linear regression is the Mean Squared Error (MSE), which measures the average squared difference between the actual and predicted values:
\begin{equation}
    \bar{L}(\theta) = \frac{1}{n} \sum_{i=1}^{n} (y_i - f(x_i; \theta))^2
\end{equation}
Minimizing the MSE ensures that the predicted values are as close as possible to the actual values.

\textbf{Example:} Given true values $y = [65, 80, 75]$ and predictions $\hat{y} = [60, 85, 70]$, the MSE would be:
\begin{equation}
    MSE = \frac{1}{3} ((65-60)^2 + (80-85)^2 + (75-70)^2) = 25
\end{equation}

\section{Logistic Regression}
Logistic regression is a classification algorithm used to predict binary outcomes by modeling the probability that a given input belongs to a particular class.

\subsection{Prediction Formula}
The logistic regression model computes a linear predictor:
\begin{equation}
    z(x; \theta) = \theta_0 + \theta_1 x_1 + \cdots + \theta_p x_p
\end{equation}

The probability of the positive class is then obtained using the sigmoid function:
\begin{equation}
    p(x; \theta) = \frac{\exp(z(x; \theta))}{1 + \exp(z(x; \theta))}
\end{equation}

\textbf{Example:} If the model predicts $z(x) = 2$, then the probability of the positive class is:
\begin{equation}
    p(x) = \frac{\exp(2)}{1 + \exp(2)} \approx 0.88
\end{equation}

\section{Interpretation}
Interpreting regression models is crucial for understanding the relationships between predictors and the outcome variable.

\subsection{Linear Regression Coefficients}
\begin{itemize}
    \item Each coefficient represents the expected change in the response variable for a one-unit increase in the corresponding predictor variable, holding all others constant.
    \item A positive coefficient indicates a positive association, while a negative coefficient suggests a negative relationship.
\end{itemize}

\textbf{Example:} In a model predicting house prices, a coefficient of 50000 for the number of rooms means that each additional room increases the predicted price by 50000.

\section{Variable Selection}
Variable selection helps to improve model interpretability and prevent overfitting by retaining only relevant predictors.

\subsection{Akaike Information Criterion (AIC)}
AIC is used to balance model complexity and goodness of fit:
\begin{equation}
    AIC = -2\hat{\ell} + 2k
\end{equation}
where:
\begin{itemize}
    \item $\hat{\ell}$ is the maximum log-likelihood,
    \item $k$ is the number of parameters.
\end{itemize}

\textbf{Example:} Comparing two models with different numbers of features, the one with the lower AIC is considered better in terms of complexity vs fit.

\end{document}
